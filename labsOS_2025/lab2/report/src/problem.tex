\section{Условие:}

Составить программу, которая реализует применение медианного фильтра к матрице произвольного размера с использованием многопоточности. Ограничение максимального количества потоков, работающих в один момент времени, должно быть задано ключом запуска вашей программы. Медианный фильтр применяется к каждому элементу матрицы в окрестности заданного размера (окно N×N, где N - нечетное число). Фильтр может применяться многократно (K итераций).
Так же необходимо уметь продемонстрировать количество потоков, используемое вашей программой с помощью стандартных средств операционной системы.

{\bfseries Цель работы:} 

Приобретение практических навыков в: управлении потоками в ОС; обеспечении синхронизации между потоками.

{\bfseries Задание:}

Наложить K раз медианный фильтр на матрицу, состоящую из целых чисел. Размер окна задается пользователем.

{\bfseries Вариант:} 11


