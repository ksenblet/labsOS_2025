\section{Описание программы}
\section{Метод решения}
Данная программа реализует многопроцессную обработку текстовых данных с использованием отображаемых файлов (memory-mapped files) для межпроцессного взаимодействия. Родительский процесс читает строки из стандартного ввода и направляет их через цепочку дочерних процессов, каждый из которых выполняет преобразование данных. Взаимодействие организовано через разделяемую память, что обеспечивает высокую производительность при обмене данными.

{\bfseries Основные компоненты:}

Parent - управляет созданием областей разделяемой памяти, запуском дочерних процессов, принимает пользовательский ввод и выводит конечный результат;

Child1 - приводит текст к нижнему регистру;

Child2 - удаляет все задвоенные пробелы;

Разделение системных вызовов в отдельную библиотеку systemCall.

\section{Описание программы}

{\bfseries Структура проекта:}

lab3/
        
\hspace{3em}include/
    
\hspace{6em}systemCall.h    // Заголовочный файл библиотеки
         
\hspace{3em}src/
    
\hspace{6em}systemCall.cpp  // Реализация системных функций
        
\hspace{6em}parent.cpp         // Родительский процесс
        
\hspace{6em}child1.cpp         // Дочерний процесс 1 (нижний регистр)
        
\hspace{6em}child2.cpp         // Дочерний процесс 2 (удаление пробелов)
        
\hspace{3em}CMakeLists.txt

{\bfseries Основные типы данных:}

1.Структура mmfT (memory-mapped file)

Содержит дескриптор области памяти, указатель на данные и размер области

На Windows использует HANDLE, на Linux - файловые дескрипторы

Позволяет организовать разделяемый доступ к памяти между процессами

2.Структура process (информация о процессе)

Хранит идентификатор запущенного процесса

На Windows содержит подробную информацию о процессе, на Linux - просто номер процесса (PID)

Содержит флаг is-valid, который показывает, работает ли процесс корректно

3.Строки std::string

4.Логические флаги (bool)

{\bfseries Принцип работы с типами данных:}

Программа создает несколько областей разделяемой памяти mmfT, через которые передаются строки std::string. Каждый дочерний процесс управляется через свою структуру process, а логические флаги следят за тем, чтобы вся система работала без ошибок. Данные передаются через общую память, что исключает необходимость сериализации и копирования.

{\bfseries Основные функции программы:}

MMFCreate() - создает новую область разделяемой памяти

MMFOpen() - открывает существующую область разделяемой памяти

MMFClose() - закрывает область памяти, освобождая ресурсы

WriteToMMF() - записывает строку в разделяемую память

ReadFromMMF() - читает строку из разделяемой памяти

ClearMMF() - очищает область памяти

ProcessCreateWithMMF() - запускает дочерний процесс с передачей имен MMF

ProcessTerminate() - принудительно завершает процесс

{\bfseries Используемые системные вызовы:}

Для Windows:

CreateFileMappingA() - создание объекта проекции файла;

MapViewOfFile() - отображение файла в память;

OpenFileMappingA() - открытие существующего объекта проекции;

UnmapViewOfFile() - отмена отображения памяти;

CreateProcessA() - создание процесса;

CloseHandle() - закрытие дескриптора;

TerminateProcess() - принудительное завершение;

Для Linux:

shm open() - создание/открытие разделяемой памяти;

mmap() - отображение памяти в адресное пространство;

munmap() - отмена отображения памяти;

ftruncate() - установка размера разделяемой памяти;

fork() - создание процесса;

exec() - загрузка новой программы;

close() - закрытие дескриптора;

kill() - отправка сигнала процессу;
