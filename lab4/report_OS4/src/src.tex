\section{Описание программы}
\section{Метод решения}
Данная программа реализует два различных способа использования динамических библиотек: статическую линковку на этапе компиляции и динамическую загрузку во время выполнения. Разработаны две динамические библиотеки с различными алгоритмами вычисления числа Пи и сортировки массивов, которые используются двумя тестовыми программами для демонстрации различий в подходах к связыванию кода.

{\bfseries Основные компоненты:}

lib1.dll - первая динамическая библиотека, реализующая вычисление числа Пи через ряд Лейбница и сортировку массива пузырьковым методом;

lib2.dll - вторая динамическая библиотека, реализующая вычисление числа Пи через формулу Валлиса и сортировку массива быстрым методом Хоара;

Program1 - программа со статической линковкой, использующая одну из библиотек на этапе компиляции;

Program2 - программа с динамической загрузкой, способная переключаться между библиотеками во время выполнения;

DynamicLoader - модуль, инкапсулирующий системные вызовы для работы с динамическими библиотеками.

\section{Описание программы}

{\bfseries Структура проекта:}

lab4/

\hspace{3em}include/

\hspace{6em}PiCalc.hpp // Абстрактный класс для вычисления числа Пи

\hspace{6em}Sorter.hpp // Абстрактный класс для сортировки массивов

\hspace{3em}lib1/

\hspace{6em}PiLeibniz.cpp // Реализация вычисления Пи методом Лейбница

\hspace{6em}BubbleSort.cpp // Реализация пузырьковой сортировки

\hspace{3em}lib2/

\hspace{6em}PiWallis.cpp // Реализация вычисления Пи методом Валлиса

\hspace{6em}QuickSort.cpp // Реализация быстрой сортировки Хоара

\hspace{3em}prog1/

\hspace{6em}main.cpp // Программа 1 (статическая линковка)

\hspace{3em}prog2/

\hspace{6em}main.cpp // Программа 2 (динамическая загрузка)

\hspace{6em}DynamicLoader.hpp // Заголовочный файл загрузчика библиотек

\hspace{6em}DynamicLoader.cpp // Реализация загрузчика библиотек

\hspace{3em}CMakeLists.txt

{\bfseries Основные типы данных:}

\textbf{Абстрактный класс PiCalc} - определяет интерфейс для вычисления числа Пи:

\textbf{Абстрактный класс Sorter} - определяет интерфейс для сортировки массивов:

\textbf{Конкретные реализации} (наследники абстрактных классов):

\texttt{LeibnizPi} - вычисление числа Пи по формуле Лейбница

\texttt{WallisPi} - вычисление числа Пи по формуле Валлиса

\texttt{BubbleSort} - сортировка пузырьковым методом

\texttt{QuickSort} - сортировка методом Хоара

\textbf{Класс DynamicLoader} - инкапсулирует работу с системными вызовами:

{\bfseries Принцип работы с типами данных:}

Программа использует полиморфизм через абстрактные классы для обеспечения единого интерфейса к различным реализациям алгоритмов. Указатели на функции DLL позволяют динамически загружать и использовать реализации без перекомпиляции основной программы. При статической линковке реализации фиксируются на этапе сборки, при динамической - могут изменяться во время выполнения.

{\bfseries Основные функции программы:}

\textbf{В динамических библиотеках:}
\begin{itemize}
\item \texttt{CreatePiCalculator()} - фабричный метод для создания объекта вычисления Пи
\item \texttt{CreateArraySorter()} - фабричный метод для создания объекта сортировки
\item \texttt{DeletePiCalculator()} - освобождение памяти объекта вычисления Пи
\item \texttt{DeleteArraySorter()} - освобождение памяти объекта сортировки
\end{itemize}

\textbf{В классе DynamicLoader:}
\begin{itemize}
\item \texttt{load()} - загрузка динамической библиотеки в память
\item \texttt{unload()} - выгрузка библиотеки из памяти
\item \texttt{switchLib()} - переключение между библиотеками
\item \texttt{getPiCalculator()} - получение объекта для вычисления Пи
\item \texttt{getArraySorter()} - получение объекта для сортировки
\end{itemize}

{\bfseries Используемые системные вызовы (инкапсулированы в DynamicLoader):}

Для Windows:
\begin{itemize}
\item \texttt{LoadLibraryA()} - загрузка динамической библиотеки в память процесса
\item \texttt{GetProcAddress()} - получение адреса экспортируемой функции по имени
\item \texttt{FreeLibrary()} - выгрузка библиотеки из памяти и уменьшение счетчика ссылок
\end{itemize}

{\bfseries Принцип работы программы:}

1. \textbf{Program1 (статическая линковка):}
   - Библиотека линкуется на этапе компиляции через CMake (команда \verb|target_link_libraries|)
   
   - Программа имеет прямой доступ к функциям библиотеки
   
   - Реализация фиксирована и не может быть изменена без перекомпиляции
   
   - Обработка команд пользователя происходит напрямую через вызовы методов объектов
   

2. \textbf{Program2 (динамическая загрузка):}
   - Библиотека загружается во время выполнения через \texttt{DynamicLoader}
   
   - Программа получает указатели на функции через \texttt{GetProcAddress()}
   
   - Возможно переключение между библиотеками командой "0"
   
   - Обработка команд осуществляется через методы \texttt{DynamicLoader}
   

При вводе команды "0" в Program2 вызывается метод \texttt{switchLib()} класса \texttt{DynamicLoader}, который:

Выгружает текущую библиотеку (\texttt{FreeLibrary()})

Загружает альтернативную библиотеку (\texttt{LoadLibraryA()})

Получает адреса функций из новой библиотеки (\texttt{GetProcAddress()})

Обновляет внутренние указатели на фабричные методы