\section{Условие:}

Создать 2 динамические библиотеки, реализующие одинаковые функции различными алгоритмами: рассчет значения числа Пи при заданной длине ряда (K), сигнатура: float Pi(int K), реализация 1: ряд Лейбница, реализация 2: формула Валлиса; сортировка целочисленного массива, сигнатура: Int * Sort(int * array), реализация 1: Пузырьковая сортировка, реализация 2: сортировка Хоара. Библиотеки должны использоваться двумя тестовыми программами разными способами: через статическую линковку на этапе компиляции и через динамическую загрузку во время выполнения.

{\bfseries Цель работы:} 

Приобретение практических навыков в: создании динамических библиотек; создании программ, использующих функции динамических библиотек двумя различными способами; анализе преимуществ и недостатков статической и динамической линковки.

{\bfseries Задание:}

1 функция: рассчет значения числа Пи при заданной длине ряда (K), сигнатура: float Pi(int K), реализация 1: ряд Лейбница, реализация 2: формула Валлиса; 

2 функция: сортировка целочисленного массива, сигнатура: Int * Sort(int * array), реализация 1: Пузырьковая сортировка, реализация 2: сортировка Хоара.

{\bfseries Вариант:} 30


