\section{Метод решения}
Данная программа реализует многопоточную обработку матричных данных с использованием потоков операционной системы Windows. Главный процесс создает N рабочих потоков для параллельного применения медианного фильтра к матрице, используя семафоры для синхронизации доступа к общим данным.

{\bfseries Основные компоненты:}

\texttt{main.cpp} - главный файл программы, управляет вводом параметров, созданием потоков, координацией работы и выводом результатов;
\texttt{matrix.h / matrix.cpp} - реализация класса для работы с матрицами (хранение, доступ, применение фильтра);
\texttt{threads.h / threads.cpp} - реализация управления потоками и синхронизации;
\texttt{CMakeLists.txt} - файл конфигурации системы сборки CMake.

\section{Описание программы}

{\bfseries Структура проекта:}

lab1/

\hspace{3em}report/
    
\hspace{6em}...
        
\hspace{3em}include/
    
\hspace{6em}matrix.h    // Заголовочный файл класса Matrix

\hspace{6em}threads.h    // Заголовочный файл класса потоков
         
\hspace{3em}src/
    
\hspace{6em}matrix.cpp  // Реализация класса Matrix
        
\hspace{6em}threads.cpp         // Реализация управления потоками
        
\hspace{3em}CMakeLists.txt

\hspace{3em}main.cpp    

{\bfseries Основные типы данных:}

1.Класс Matrix

Хранит данные в виде вектора векторов \texttt{std::vector<std::vector<int>>}

Содержит размеры матрицы: количество строк и столбцов

Предоставляет методы доступа к элементам и применения медианного фильтра

2.Класс Semaphore

Использует дескриптор Windows \texttt{HANDLE}
 
Реализует операции \texttt{wait()} и \texttt{signal()} для синхронизации

Обеспечивает взаимное исключение при доступе к общим данным

3.Структура ThreadData (данные потока)

Содержит указатели на входную и выходную матрицы

Хранит параметры фильтра (размер окна, количество итераций)

Содержит синхронизационные объекты и счетчики для распределения работы

4.Класс ThreadControl (управление потоками)

Управляет созданием и завершением потоков

Распределяет работу между потоками

Координирует доступ к общим ресурсам через семафоры

{\bfseries Принцип работы с типами данных:}

Программа создает объект \texttt{Matrix} для хранения входных и выходных данных. Объект \texttt{ThreadControl} управляет несколькими потоками, каждый из которых получает копию структуры \texttt{ThreadData} с указателями на матрицы и параметрами обработки. Семафоры обеспечивают корректный доступ к общему счетчику строк.

{\bfseries Основные функции программы:}

\begin{itemize}
    \item \texttt{Matrix::applyMedianFilter()} - применяет медианный фильтр к элементу матрицы;
    \item \texttt{ThreadControl::applyMedianFilter()} - организует многопоточную обработку матрицы;
    \item \texttt{ProcessMatrixRegion()} - функция, выполняемая каждым рабочим потоком;
    \item \texttt{ThreadCreate()} / \texttt{ThreadJoin()} - создание и ожидание завершения потоков;
    \item \texttt{Semaphore::wait()} / \texttt{Semaphore::signal()} - операции с семафором.
\end{itemize}

{\bfseries Используемые системные вызовы:}

\begin{itemize}
    \item \texttt{CreateThread()} - создание потока;
    \item \texttt{WaitForSingleObject()} - ожидание объекта (потока, семафора);
    \item \texttt{CreateSemaphore()} - создание семафора;
    \item \texttt{ReleaseSemaphore()} - увеличение счетчика семафора;
    \item \texttt{CloseHandle()} - закрытие дескрипторов.
\end{itemize}

{\bfseries Алгоритм распределения работы:}

Программа использует динамическое распределение строк матрицы между потоками. Общий счетчик \texttt{nextRow} защищается семафором и указывает на следующую обрабатываемую строку. Каждый поток берет следующую доступную строку, обрабатывает ее и повторяет до тех пор, пока все строки не будут обработаны.