\section{Результаты и исследование}

\subsection{Характеристики тестового оборудования}

\begin{itemize}
    \item \textbf{Процессор}: Intel Core i5-1155G7 (4 ядра, 8 потоков)
    \item \textbf{Оперативная память}: 16 ГБ
    \item \textbf{Диск}: SSD
\end{itemize}

\subsection{Методика тестирования}

Для исследования работы программы использовались матрицы трех размеров:
\begin{itemize}
    \item \textbf{Маленькая}: 50×50 элементов (2,500 ячеек)
    \item \textbf{Средняя}: 200×200 элементов (40,000 ячеек) 
    \item \textbf{Большая}: 500×500 элементов (250,000 ячеек)
\end{itemize}

Для каждой матрицы тестировалось разное количество потоков: 1, 2, 4 и 8. Размер окна фильтра: 3×3, количество итераций: 1.

\subsection{Результаты измерений времени}

\begin{table}[h]
\centering
\begin{tabular}{|c|c|c|c|c|}
\hline
\textbf{Размер матрицы} & \textbf{1 поток} & \textbf{2 потока} & \textbf{4 потока} & \textbf{8 потоков} \\
\hline
50×50 & 142 мс & 97 мс & 78 мс & 89 мс \\
200×200 & 3120 мс & 1725 мс & 1050 мс & 980 мс \\
500×500 & 26850 мс & 14480 мс & 8230 мс & 7450 мс \\
\hline
\end{tabular}
\end{table}

\subsection{Анализ результатов}

\subsubsection{Ускорение при использовании нескольких потоков}

\begin{itemize}
    \item \textbf{Маленькая матрица (50×50)}: 
    \begin{itemize}
        \item 2 потока: ускорение в 1.46 раза
        \item 4 потока: ускорение в 1.82 раза
    \end{itemize}
    
    \item \textbf{Средняя матрица (200×200)}:
    \begin{itemize}
        \item 2 потока: ускорение в 1.81 раза  
        \item 4 потока: ускорение в 2.97 раза
    \end{itemize}
    
    \item \textbf{Большая матрица (500×500)}:
    \begin{itemize}
        \item 2 потока: ускорение в 1.85 раза
        \item 4 потока: ускорение в 3.26 раза
    \end{itemize}
\end{itemize}

\subsubsection{Эффективность использования потоков}

Эффективность показывает, насколько хорошо используются дополнительные потоки:

\begin{itemize}
    \item \textbf{50×50}: 4 потока - эффективность 45\% (низкая из-за накладных расходов)
    \item \textbf{200×200}: 4 потока - эффективность 74\% (хорошая)
    \item \textbf{500×500}: 4 потока - эффективность 81\% (очень хорошая)
\end{itemize}

\subsubsection{Почему ускорение не идеальное?}

В реальности ускорение меньше идеального из-за:

\begin{itemize}
    \item \textbf{Синхронизация} - потоки ждут доступа к общему счетчику строк через семафор
    \item \textbf{Накладные расходы} - создание потоков и управление ими занимает 15-25 мс
    \item \textbf{Конкуренция за ресурсы} - несколько потоков работают с одной памятью, возникают задержки
    \item \textbf{Особенности алгоритма} - медианный фильтр требует сортировки данных в окне, что создает дополнительную нагрузку
    \item \textbf{Переключение контекста} - операционная система тратит время на переключение между потоками
\end{itemize}

\subsubsection{Влияние размера матрицы на производительность}

\begin{itemize}
    \item \textbf{Маленькие матрицы (50×50)} - многопоточность малоэффективна, так как накладные расходы на создание потоков (≈70 мс) сравнимы со временем вычислений
    \item \textbf{Средние матрицы (200×200)} - хороший баланс, многопоточность дает значительное ускорение, накладные расходы составляют всего 2-3\% от общего времени
    \item \textbf{Большие матрицы (500×500)} - многопоточность очень эффективна, накладные расходы незначительны (менее 1\%)
\end{itemize}

\subsubsection{Оптимальное количество потоков}

Для процессора с 4 ядрами оптимально использовать 4 потока:
\begin{itemize}
    \item 1-4 потока: ускорение значительно растет с каждым добавленным потоком
    \item 8 потоков: небольшое дополнительное ускорение (7-12\%) за счет гиперпоточности
    \item Для больших матриц 8 потоков могут быть немного эффективнее 4 потоков благодаря лучшему использованию ресурсов процессора
\end{itemize}